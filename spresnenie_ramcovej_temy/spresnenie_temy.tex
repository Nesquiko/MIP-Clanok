\documentclass[11pt,twoside,slovak,a4paper]{article}

\usepackage[slovak]{babel}
\usepackage[IL2]{fontenc} % lepšia sadzba písmena Ľ než v T1
\usepackage[utf8]{inputenc}
\usepackage{hyperref} % odkazy v texte budú aktívne (pri niektorých triedach dokumentov spôsobuje posun textu)

\usepackage{cite}

\oddsidemargin=0cm % text na neparnych stranach bude vycentrovany lebo margin = 0
\textwidth=16.5cm % sirka textu na strane
\textheight=23cm % vyska textovej casti na strane

\title{Využitie statickej analýzy kódu pri vývoji softvéru}

\author{Lukáš Častven\\[2pt]
	{\small Slovenská technická univerzita v Bratislave}\\
	{\small Fakulta informatiky a informačných technológií}\\
	{\small \texttt{xcastven@stuba.sk}}
	}

\date{\small 8. október 2021} 



\begin{document}

\maketitle
\thispagestyle{empty} % odstrani cislo na spodku strany

\begin{abstract}
	Statická analýza je proces, pri ktorom je počítačový kód zanalyzovaný bez samotného spúšťania kódu.
	Po tejto procedúre, sú programátorovi prezentované nájdené chyby, ich možný spôsob opravy a aj varovania
	o menej závažných nedostatkoch a ich riešenia. Pomocou tejto metódy dokážeme v celom analyzovanom projekte
	zlepšiť kvalitu kódu a udržať konzistentný štýl, ktorý taktiež spĺňa osvedčené postupy pri vývoji softvéru.
	Veľkou výhodou je tiež urýchlenie hľadania chýb a softvérových defektov v porovnaní s manuálnou kontrolou.
	V tomto článku pochopíme, prečo developeri používajú nástroje statickej analýzy, ako ich používajú
	na opravu a zlepšenie kódu a ako ich implementujú do ich pracovného prostredia.
\end{abstract}

\paragraph{Zdroj}
Why don't software developers use static analysis tools to find bugs?
najdete \textbf{\underline{\href{https://ieeexplore.ieee.org/document/6606613}{tu}}}.
\footnote{https://ieeexplore.ieee.org/document/6606613}

\end{document}
